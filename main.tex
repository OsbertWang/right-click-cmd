% !TeX document-id = {663102d9-8508-4921-875e-4d74aaf2cd3e}
% !TeX TXS-program:compile = txs:///latexmk/{}[-pdfxe -synctex=1 -interaction=nonstopmode -file-line-error -silent %]
\documentclass{ctexart}
\usepackage[a4paper,margin=1in]{geometry}
\usepackage{listings}
\usepackage{xcolor}
\usepackage{fontawesome}
\usepackage{hyperref}
\usepackage{cleveref}
\usepackage[os=win,hyperrefcolorlinks]{menukeys}

\setfontfamily{\FA}{[FontAwesome.otf]}

\renewmenumacro{\menu}[>]{angularmenus}
\renewmenumacro{\keys}[+]{shadowedroundedkeys}
\renewmenumacro{\directory}[/]{paths}
\makeatletter
\tw@make@key@box{OS@mac}{\faApple}
\tw@make@key@box{OS@win}{\faWindows}
\tw@make@key@macro*{\OS}
\makeatother

\lstset{
  backgroundcolor = \color{lightgray!30},
  keywordstyle = \color{blue},
  stringstyle = \color{purple!40},
  basicstyle = {\small\ttfamily},
  breaklines = true,
  tabsize = 4,
  gobble = 2,
  numbers = left,
  numberstyle = \tiny,
  emphstyle={\color{blue}\small\ttfamily}%
}

\title{右击直接打开 \textsf{cmd} 窗口}
\author{啸行\thanks{ranwang.osbert@outlook.com}}

\begin{document}
\maketitle
\begin{abstract}
  Windows 10 系统中,右击可打开 \textsf{Powershell} 窗口\footnote{本文均省略需要 \keys{\shift} 辅助的描述}。
  修改注册表后,用户可打开 \textsf{cmd} 窗口而非 \textsf{Powershell} 窗口。
  本文将记录这种修改方法。
  强烈建议用户在修改注册表前先进行备份。
\end{abstract}

\section{右击文件夹图标打开 \textsf{cmd} 窗口}

\begin{enumerate}
  \item 使用 \keys{\OSwin + R} 打开“运行”窗口,输入 \texttt{regedit} 并且点击 \menu{确定},进入 \menu{注册表编辑器};
  \item 在左侧找到 \directory{HKEY\_CLASSES\_ROOT/Directory/shell/cmd};
  \item 右键 \menu{cmd} 文件夹,进入 \menu{权限};
  \item 点击 \menu{高级},在 \menu{所有者} 右侧点击 \menu{更改};
  \item 在 \menu{输入要选择的对象名称} 处,输入计算机用户名\footnote{在 \keys{\OSwin} 中的\directory{设置 / 账户}内可查计算机用户名},\menu{检查名称} 无误后,\menu{确定};
  \item 勾选 \menu{替换子容器和对象的所有者} 后 \menu{确定};
  \item 在 \menu{组或用户名} 中选择 \menu{Administrators},在下方 \menu{权限} 处 \menu{允许} 完全控制后 \menu{确定};
  \item 在右边找到 \menu{HideBasedOnVelocityId},右键选择 \menu{重命名},改为 \menu{ShowBasedOnVelocityId};
  \item 关闭 \menu{注册表编辑器}。
\end{enumerate}
至此,用户右击文件夹图标时便可出现 \menu{在此处打开命令窗口}。

\section{右击文件夹背景打开 \textsf{cmd} 窗口}

\begin{enumerate}
  \item 使用 \keys{\OSwin + R} 打开“运行”窗口,输入 \texttt{regedit} 并且点击 \menu{确定},进入 \menu{注册表编辑器};
  \item 在左侧找到 \directory{HKEY\_CLASSES\_ROOT/Directory/Background/shell/cmd};
  \item 右键 \menu{cmd} 文件夹,进入 \menu{权限};
  \item 点击 \menu{高级},在 \menu{所有者} 右侧点击 \menu{更改};
  \item 在 \menu{输入要选择的对象名称} 处,输入计算机用户名,\menu{检查名称} 无误后,\menu{确定};
  \item 勾选 \menu{替换子容器和对象的所有者} 后 \menu{确定};
  \item 在 \menu{组或用户名} 中选择 \menu{Administrators},在下方 \menu{权限} 处 \menu{允许} 完全控制后 \menu{确定};
  \item 在右边找到 \menu{HideBasedOnVelocityId},右键选择 \menu{重命名},改为 \menu{ShowBasedOnVelocityId};
  \item 关闭 \menu{注册表编辑器}。
\end{enumerate}
至此,用户右击文件夹背景时便可出现 \menu{在此处打开命令窗口}。

\section{有用的命令行代码}

将文件夹内所有 \texttt{png} 文件转换为 \texttt{pdf} 文件的代码

\begin{lstlisting}[language = bash]
  echo off
  for /f "delims=" %%i in ('dir /b *.png') do (
  bitmap2eps %%i %%~ni.eps
  epstopdf %%~ni.eps %%~ni.pdf
  )
  del *.eps
\end{lstlisting}

检查文件的哈希值代码,这里给出的是 \verb|md5|,其他可以考虑 \verb|md2|,\verb|md4|,\verb|SHA1|,\verb|SHA256|,\verb|SHA384|,\verb|SHA512|。

\begin{lstlisting}[language = bash]
  certutil -hashfile texlive2019.iso md5
\end{lstlisting}

\section{后记}

本文概述了显示 \textsf{cmd} 的方式。
如果要隐藏 \textsf{Powershell},可以仿照前述方法,将 \menu{ShowBasedOnVelocityId} 重命名为 \menu{HideBasedOnVelocityId} 即可。

\end{document}